\documentclass[a4paper,12pt]{report}
\usepackage{geometry}
\usepackage{hyperref}
\usepackage{xcolor}
\usepackage{listings}
\geometry{margin=1in}
\title{ARG Detection and Synteny Analysis Workflow Documentation}
\author{Gerald Amiel Ballena}
\date{\today}

\lstset{
	basicstyle=\ttfamily\small,
	breaklines=true,            % Enable line breaking
	breakatwhitespace=false,    % Allow breaking anywhere
	frame=single,               % Add a single-line frame
	backgroundcolor=\color{gray!10},
	showstringspaces=false,
	keywordstyle=\color{blue},
	commentstyle=\color{green!60!black},
	stringstyle=\color{orange}
}

\begin{document}
	
	\maketitle
	
	\section*{Overview}
	This document details the workflow implemented for analyzing plasmid data downloaded from the IMG/PR database. The workflow focuses on identifying plasmids with antimicrobial resistance genes (ARGs) and performing synteny analysis on ARG-positive plasmids. These analyses are critical in plasmid research as ARGs play a pivotal role in the spread of antibiotic resistance, which poses a significant public health threat. By combining ARG detection with synteny analysis, this workflow allows researchers to investigate the genetic context and evolutionary relationships of ARG-containing plasmids, providing insights into their potential mobility and functionality.
	
	\section*{Data Files}
	The following files were provided in the \textbf{IMGPR} folder:
	\begin{itemize}
		\item \textbf{IMGPR\_plasmid\_data.tsv}: Metadata related to plasmids.
		\item \textbf{IMGPR\_nucl.fna.gz}: Compressed nucleotide sequences.
		\item \textbf{IMGPR\_prot.faa.gz}: Compressed protein sequences.
		\item \textbf{README.md}: Documentation accompanying the dataset.
	\end{itemize}
	
	\section*{Steps Performed}
	
	\subsection*{Decompress}
	The nucleotide and protein sequences were decompressed and cleaned using the following commands:
	\begin{lstlisting}[language=bash]
		gunzip IMGPR_nucl.fna.gz
	\end{lstlisting}
	
	\subsection*{ARG Detection}
	ARG detection was performed directly using \textbf{RGI} with the CARD database:
	\begin{lstlisting}[language=bash]

		rgi load -i card.json --card_annotation protein_fasta_protein_variant_model.fasta --local
		
		rgi database --local -v
		
		mkdir results 
		
		chmod u+rwx ~/Desktop/*/results
		chmod u+r IMGPR_nucl.fna
		
		 rgi main -i IMGPR_nucl.fna -o results/ARGs_rgi_output -t contig --local --clean -n 4
		


	\end{lstlisting}
	
	\subsection*{ARG-Positive Plasmid Extraction}
	Plasmids with detected ARGs were identified and their sequences extracted:
	\begin{lstlisting}[language=bash]
		grep ">" results/ARGs_rgi_output.txt | cut -d' ' -f1 > results/arg_positive_ids.txt
		seqtk subseq cleaned_headers.fasta results/arg_positive_ids.txt > results/arg_positive_plasmids.fna
	\end{lstlisting}
	
	\subsection*{Synteny Analysis}
	Synteny analysis was performed on ARG-positive plasmids using \textbf{MUMmer}:
	\begin{lstlisting}[language=bash]
		nucmer --prefix results/synteny/arg_positive_synteny \
		results/arg_positive_plasmids.fna \
		results/arg_positive_plasmids.fna
		show-coords -rcl results/synteny/arg_positive_synteny.delta > results/synteny/arg_positive_synteny.coords
	\end{lstlisting}
	
	\section*{Outputs}
	The following results were generated:
	\begin{itemize}
		\item \textbf{plasmids\_metadata.sqlite}: SQLite database containing plasmid metadata.
		\item \textbf{results/ARGs\_rgi\_output.txt}: ARG detection results from RGI.
		\item \textbf{results/arg\_positive\_plasmids.fna}: FASTA file of ARG-positive plasmids.
		\item \textbf{results/synteny/}: Synteny alignment and coordinates for ARG-positive plasmids.
	\end{itemize}
	
	\section*{Future Directions}
	\begin{itemize}
		\item Expanded synteny analysis using progressiveMauve.
		\item Comparative functional analysis of ARG-positive vs ARG-negative plasmids.
		\item Visualization of synteny and ARG locations using Circos or Artemis.
	\end{itemize}
	
	\section*{Acknowledgments}
	Data sourced from the IMG/PR database.
	
\end{document}
